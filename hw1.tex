
\documentclass[a4paper,12pt,oneside,article]{memoir}
\setlrmarginsandblock{3cm}{*}{1}
\setulmarginsandblock{3cm}{*}{1}
\checkandfixthelayout[nearest]

\usepackage[utf8]{inputenc}
\usepackage[russian]{babel}      
\usepackage{amsmath, amssymb, bm, mathtools, mathdots}

\usepackage[normalem]{ulem}

\makepagestyle{style}
\copypagestyle{style}{empty}
\makeoddhead{style}{Жидович Антон \\2 курс, 7 группа}{\quad}{}
\makeheadrule{style}{\textwidth}{\normalrulethickness}
\pagestyle{style} 

\usepackage{enumerate}

\begin{document}
\chapter*{Задание} 
Вычислить интеграл
\begin{equation*}
    \oint\limits_C\frac{e^z}{(z-i)^2(z+2)}dz
\end{equation*}
в следующих случаях контура:
\begin{enumerate}[1)]
    \item $|z-i|=2$
    \item $|z+2-i|=3$
\end{enumerate}

\chapter*{Решение}
1) В круг $|z-i|<2$ попадает точка $z=i$. Запишем подынтегральную функцию в виде  {\Large $\frac{\frac{e^z}{z+2}}{(z-i)^2}$}  и используем формулу кратности 

\begin{equation*}
    \oint\limits_C\frac{f(z)}{(z-a)^n}dz=\frac{2\pi i}{(n-1)!}f^{(n-1)}(a)
\end{equation*}
для корня $a=i$ кратности 2. Вычисляем интеграл:
\begin{equation*}
    \oint\limits_C\frac{e^z}{(z-i)^2(z+2)}dz=2\pi i\Big(\frac{e^z}{z+2}\Big)'\bigg|_{z=i}= \frac{2\pi i (1+i)}{(2+i)^2}e^i.
\end{equation*}
2) В круг $|z+2-i|<3$ входят обе точки $z_1=i, z_2=-2$. Решаем в соответствии с формулой 
\begin{equation*}
   \oint\limits_Cf(z)dz=\oint\limits_{C_1}f(z)dz+\oint\limits_{C_2} f(z)dz,
\end{equation*}
где каждый из $C_1$ и $C_2$ охватывает только одну из точек. В частности, в качестве контура $C_1$ можно взять окружность из предыдущего пункта, $C_2$ - окружность $|z+2+i|=2$.
\begin{equation*}
    \oint\limits_C\frac{e^z}{(z-i)^2(z+2)}dz
    =2 \pi i \frac{e^{-2}}{(2+i)^2}+2\pi i \frac{1+i}{(2+i)^2}e^i
    =\frac{2 \pi i}{(2+i)^2}\Big(e^{-2}+e^i(1+i)\Big).
\end{equation*}
\end{document}